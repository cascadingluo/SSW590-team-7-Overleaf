\chapter{Project Proposal \\
\small{\textit{-- Luo Xu, Annanya Jain, Gavin Lam}}
\index{Project Proposal} 
\index{Chapter!Project Proposal}
\label{Chapter::Project Proposal}}

Our project is an AI Health Voice Assistant named AVA, short for Artificial Voice Assistant. Users will be able to log in and create an account and chat with AVA. This project aims to support users in tracking emotions, managing reminders, and accessing mental wellness resources. Unlike traditional chatbots, our assistant goes beyond simple question–answer interaction by integrating:

\begin{itemize}
\item Agentic AI workflows, enabling the assistant to interpret user intent, plan actions, and decide between generating responses, retrieving wellness exercises, or scheduling reminders.

\item DevSecOps practices to make it scalable, and secure from development to deployment.
The end goal is a functional prototype that not only demonstrates AI-driven health support but also serves as a practical use of modern DevOps pipelines and monitoring
\end{itemize} 
This project will be an enhancement of our previously created project for CS555 course. Since then we have grown a lot in terms of knowledge and skill sets and we would like to improve it to have agentic powers 
to better aid users. Here is the link for the repository for the project: 
\url{https://github.com/cascadingluo/SSW590-team-7-project}

\begin{longtable}{|p{4cm}|p{10cm}|}
\hline
\textbf{Tool} & \textbf{Usage} \\ \hline
Flask & Web framework. \\ \hline
MongoDB & Database for user information and chat history storage. \\ \hline
Gemini & AI API used for the chatbot. \\ \hline
GitHub & Source Control and Collaboration \\ \hline
CI/CD & Github Actions for automated testing\\ \hline
Jira & Issue tracking and Agile project management tool. \\ \hline
AWS & Pushing local Docker with AWS and deploying with App Runner. \\ \hline
TikZ & Vizualizing Architecture of our assistant. \\ \hline 
\end{longtable}
