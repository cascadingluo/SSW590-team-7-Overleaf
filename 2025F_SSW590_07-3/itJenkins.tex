\chapter{Install Jenkins and Run a Pytest \\
\small{\textit{-- Annanya Jain, Luo Xu, Gavin Lam}
\index{Overleaf} 
\index{Chapter!Install Jenkins and Run a Pytest}
\label{Chapter::Install Jenkins and Run a Pytest}}}

\section{Introduction}
% \label{sec:Install Jenkins and Run a Pytest}

This document provides detailed documentation of a CI/CD pipeline setup using Jenkins in Docker, a Python project with Pytest, and Git integration for source control.

\section{Part 1: Jenkins Setup in Docker} To set up Jenkins within a Docker container, the following \text{docker-compose.yml} file was created:


\begin{minted}{yaml}
version: '3.8'
services:
  jenkins:
    image: jenkins/jenkins:lts
    container_name: jenkins
    ports:
      - "8080:8080"
      - "50000:50000"
    volumes:
      - jenkins_home:/var/jenkins_home
      - /var/run/docker.sock:/var/run/docker.sock

volumes:
  jenkins_home:
\end{minted}

To start Jenkins using Docker Compose, run:

\begin{minted}{bash} 
$ docker compose up -d 
\end{minted}

Once the container is running, Jenkins can be accessed at \url{http://localhost:8080}. The initial admin password is retrieved with:

\begin{minted}{bash} 
$ docker exec -it jenkins cat /var/jenkins_home/secrets/initialAdminPassword 
\end{minted}

\section{Part 2: Python + Pytest Project} The following directory structure was created for the Python testing project:

\begin{minted}{bash} 
jenkins-python-pytest-demo/ 
|-- Jenkinsfile 
|-- requirements.txt 
|-- tests/ 
  -- test_sample.py 
\end{minted}

The \text{requirements.txt} file includes the following dependency:

\begin{minted}{text} 
pytest 
\end{minted}

\begin{minted}{python}
def test_addition():
    assert 1 + 1 == 2

def test_subtraction():
    assert 5 - 2 == 3

def test_failure_example():
    assert 2 * 2 == 5  # This will fail
\end{minted}

\section{Part 3: Jenkins Pipeline Configuration} A Jenkins Pipeline was configured to automate dependency installation, test execution, and reporting using the following Jenkinsfile:

\begin{minted}{groovy}
pipeline {
    agent any
    stages {
        stage('Install dependencies') {
            steps {
                sh 'python3 -m venv venv'
                sh './venv/bin/pip install -r requirements.txt'
            }
        }
        stage('Run tests') {
            steps {
                sh './venv/bin/pytest --junitxml=report.xml'
            }
        }
        stage('Publish Report') {
            steps {
                junit 'report.xml'
            }
        }
    }
}
\end{minted}

\section{Part 4: Git Integration} The project was initialized as a Git repository and pushed to GitHub using:

\begin{minted}{bash} 
$ git init 
$ git add . 
$ git commit -m "Initial commit"
$ git remote add origin https://github.com/JainAnnanya/SSW590-team-7-Jenkins.git
$ git branch -M main 
$ git push -u origin main 
\end{minted}

In Jenkins, the repository URL was added under the "Pipeline script from SCM" option, and the branch was set to \text{main} to automatically load the Jenkinsfile during job execution.

\section{Part 5: Running the Pipeline and Viewing Test Reports} Upon running the Jenkins job, the following stages were executed:

\begin{itemize} 
\item Install dependencies 
\item Run tests 
\item Publish report 
\end{itemize}

Passed test cases by running only 1st and 2nd test case
\begin{center}
  \includegraphics[width=0.5\textwidth]{png/tests_passed.png}
\end{center}

\begin{center}
  \includegraphics[width=0.9\textwidth]{png/passes_test_cases.png}
  \caption{Jenkins UI after successful job execution}
\end{center}

When running all three test cases, we can see one test case failed and two passed. 

\begin{center}
  \includegraphics[width=0.5\textwidth]{png/Showing2testspassed_and1failed.png}
\end{center}

\begin{center}
  \includegraphics[width=0.5\textwidth]{png/Showingtestfailed.png}
\end{center}

The test results were available under the "Test Result" tab in Jenkins. The failing test in the file was reported as expected. The pipeline stage view provided a graphical representation of each stage's status. 
